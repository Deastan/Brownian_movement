\section{Systeme.\+cc}
\label{_systeme_8cc_source}\index{/\+Users/burkhard/\+Dropbox/projet d'info/maximilian-\/g065/\+Simulation d'un gaz parfait/\+Systeme.\+cc@{/\+Users/burkhard/\+Dropbox/projet d'info/maximilian-\/g065/\+Simulation d'un gaz parfait/\+Systeme.\+cc}}

\begin{DoxyCode}
00001 
00009 \textcolor{preprocessor}{#include "Systeme.h"}
00010 \textcolor{preprocessor}{#include "GLNeon.h"}
00011 \textcolor{preprocessor}{#include "GLHelium.h"}
00012 \textcolor{preprocessor}{#include "GLArgon.h"}
00013 \textcolor{preprocessor}{#include "GLFluor.h"}
00014 
00015 \textcolor{preprocessor}{#ifndef WX\_PRECOMP}
00016 \textcolor{preprocessor}{#include "wx/wx.h"}
00017 \textcolor{preprocessor}{#endif}
00018 \textcolor{preprocessor}{#include "wx/glcanvas.h"} \textcolor{comment}{// Pour combiner wxWidgets et OpenGL}
00019 
00020 \textcolor{keyword}{using namespace }std;
00021 
00022 \textcolor{comment}{/*================================================================================}
00023 \textcolor{comment}{ * Definition des constructeurs et du destructeur}
00024 \textcolor{comment}{ *================================================================================*/}
00025 
00027 Systeme::Systeme()
00028  : temperature(298) \textcolor{comment}{// Température normale : 25°C}
00029 \{ initialise(); \}
00030 
00037 Systeme::Systeme(\textcolor{keywordtype}{bool} reglage)
00038  : enceinte(reglage), temperature(298)
00039 \{
00040     \textcolor{keywordflow}{do}
00041     \{
00042         temperature = wxAtoi(wxGetTextFromUser(wxT(\textcolor{stringliteral}{"Rentrez une température (entre 0 et 400 Kelvin)"}), wxT(\textcolor{stringliteral}{
      "Température"})));
00043     \} \textcolor{keywordflow}{while} (temperature <= 0 or temperature > 400);
00044     
00045     \textcolor{keywordtype}{double} nbrArgon(0);
00046     \textcolor{keywordflow}{do}
00047     \{
00048         nbrArgon = wxAtoi(wxGetTextFromUser(wxT(\textcolor{stringliteral}{"Rentrez le nombre de particules d'Argon (entre 0 et 425,
       Attention : ne pas créer plus de 425 particules en tout)"}), wxT(\textcolor{stringliteral}{"Paramétrage du nombre de particules"})));
00049     \} \textcolor{keywordflow}{while} (nbrArgon < 0 or nbrArgon > 425);
00050     
00051     \textcolor{keywordtype}{double} nbrFluor(0);
00052     \textcolor{keywordflow}{do}
00053     \{
00054         nbrFluor = wxAtoi(wxGetTextFromUser(wxT(\textcolor{stringliteral}{"Rentrez le nombre de particules de Fluor (entre 0 et 3,
       Attention : ne pas créer plus de 425 particules en tout)"}), wxT(\textcolor{stringliteral}{"Paramétrage du nombre de particules"})));
00055     \} \textcolor{keywordflow}{while} (nbrFluor < 0 or nbrFluor > 3);
00056     
00057     \textcolor{keywordtype}{double} nbrHelium(0);
00058     \textcolor{keywordflow}{do}
00059     \{
00060         nbrHelium = wxAtoi(wxGetTextFromUser(wxT(\textcolor{stringliteral}{"Rentrez le nombre de particules d'Helium (entre 0 et 425,
       Attention : ne pas créer plus de 425 particules en tout)"}), wxT(\textcolor{stringliteral}{"Paramétrage du nombre de particules"})));
00061     \} \textcolor{keywordflow}{while} (nbrHelium < 0 or nbrHelium > 425);
00062     
00063     \textcolor{keywordtype}{double} nbrNeon(0);
00064     \textcolor{keywordflow}{do}
00065     \{
00066         nbrNeon = wxAtoi(wxGetTextFromUser(wxT(\textcolor{stringliteral}{"Rentrez le nombre de particules de Néon (entre 0 et 425,
       Attention : ne pas créer plus de 425 particules en tout)"}), wxT(\textcolor{stringliteral}{"Paramétrage du nombre de particules"})));
00067     \} \textcolor{keywordflow}{while} (nbrNeon < 0 or nbrNeon > 425);
00068     
00069     initialise(nbrArgon,nbrFluor,nbrHelium,nbrNeon);
00070 \}
00071 
00079 Systeme::Systeme(\textcolor{keywordtype}{double} largeur, \textcolor{keywordtype}{double} longueur, \textcolor{keywordtype}{double} hauteur, \textcolor{keywordtype}{double} temperature)
00080  : enceinte(largeur, longueur, hauteur), temperature(temperature)
00081 \{ initialise(); \}
00082 
00084 Systeme::~Systeme()
00085 \{ supprimerParticules(); \}
00086 
00087 \textcolor{comment}{/*================================================================================}
00088 \textcolor{comment}{ * Definition des methodes}
00089 \textcolor{comment}{ *================================================================================*/}
00090 
00092 \textcolor{keywordtype}{void} Systeme::ajouterParticule(Particule* particule)
00093 \{ 
00094     \textcolor{keywordflow}{if}(particule != \textcolor{keyword}{nullptr})
00095     \{ collectionParticules.push\_back(unique\_ptr<Particule>(particule)); \}
00096 \}
00097 
00099 \textcolor{keywordtype}{void} Systeme::dessine()\textcolor{keyword}{ const}
00100 \textcolor{keyword}{}\{   
00101     glPushMatrix();
00102     glTranslated(-enceinte.getLargeur()/2,-enceinte.getLongueur()/2,-enceinte.
      getHauteur()/2);
00103     enceinte.dessine();
00104     canon.dessine();
00105     
00106     \textcolor{keywordflow}{for}(\textcolor{keyword}{auto} \textcolor{keyword}{const}& particule : collectionParticules) 
00107     \{ particule->dessine(); \}
00108     glPopMatrix();
00109 \}
00110 
00114 \textcolor{keywordtype}{bool} Systeme::supprimerParticules()
00115 \{
00116     \textcolor{keywordflow}{for}(\textcolor{keyword}{auto}& particule : collectionParticules)
00117     \{ particule.reset(); \}
00118     \textcolor{keywordflow}{return} \textcolor{keyword}{true};
00119 \}
00120 
00126 \textcolor{keywordtype}{void} Systeme::evolue(\textcolor{keywordtype}{double} dt)
00127 \{
00128     \textcolor{keywordflow}{for} (\textcolor{keyword}{auto} \textcolor{keyword}{const}& particule : collectionParticules) 
00129     \{
00130         particule->evolue(dt);
00131         
00132         particule->gere_sorties(enceinte);
00133         
00134         vector<size\_t> collisions\_possibles (determine\_collisions\_possibles(particule));
00135         \textcolor{keywordflow}{if} (collisions\_possibles.size() != 0)
00136         \{ particule->collision(*collectionParticules[choisi\_collision(collisions\_possibles)], tirage); \}
00137     \}
00138 \}
00139 
00143 vector<size\_t> Systeme::determine\_collisions\_possibles(unique\_ptr<Particule> \textcolor{keyword}{const}& p1)
00144 \{
00145     vector<size\_t> tableau;
00146     \textcolor{keywordflow}{for}(\textcolor{keywordtype}{size\_t} i(0);i<collectionParticules.size();++i)
00147     \{
00148         \textcolor{keywordflow}{if}((p1 != collectionParticules[i]) and (p1->pavageCubique(epsilon) == collectionParticules[i]->
      pavageCubique(epsilon)))
00149         \{ tableau.push\_back(i); \}
00150     \}
00151     \textcolor{keywordflow}{return} tableau;
00152 \}
00153 
00156 \textcolor{keywordtype}{size\_t} Systeme::choisi\_collision(vector<size\_t> \textcolor{keyword}{const}& collisions\_possibles)
00157 \{ \textcolor{keywordflow}{return} collisions\_possibles[tirage.uniforme_entier(0, collisions\_possibles.size() - 1)]; \}
00158   \textcolor{comment}{//le tirage uniforme d'ENTIER se faisant sur un intervalle fermé [a, b], il faut donc tirer sur [0,
       collisions\_possibles.size() - 1] pour que ce soit juste}
00159 
00164 \textcolor{keywordtype}{void} Systeme::ajouterArgon(\textcolor{keywordtype}{unsigned} \textcolor{keywordtype}{int} nbr\_argon, \textcolor{keywordtype}{bool} canon)
00165 \{
00166     GLArgon argon(Vecteur(0,0,0), Vecteur(0,0,0));
00167     \textcolor{keywordtype}{double} rayon\_argon(argon.get_rayon());
00168     \textcolor{keywordflow}{if} (nbr\_argon != 0) \{ change\_epsilon(rayon\_argon); \}
00169     \textcolor{keywordflow}{if} (canon) \{
00170         \textcolor{keywordflow}{for}(\textcolor{keywordtype}{int} i(1) ; i <= nbr\_argon ; ++i)
00171         \{
00172             \textcolor{keywordtype}{double} pas(i*rayon\_argon);
00173             ajouterParticule(\textcolor{keyword}{new} GLArgon(Vecteur(pas,pas,pas), Vecteur(250,250,250)));
00174         \}
00175     \} \textcolor{keywordflow}{else} \{
00176         \textcolor{keywordflow}{for}(\textcolor{keywordtype}{int} i(1) ; i <= nbr\_argon ; ++i)
00177         \{ ajouterParticule(\textcolor{keyword}{new} GLArgon(enceinte, tirage, temperature)); \}
00178     \}
00179 \}
00180 
00185 \textcolor{keywordtype}{void} Systeme::ajouterFluor(\textcolor{keywordtype}{unsigned} \textcolor{keywordtype}{int} nbr\_fluor, \textcolor{keywordtype}{bool} canon)
00186 \{
00187     GLFluor fluor(Vecteur(0,0,0), Vecteur(0,0,0));
00188     \textcolor{keywordtype}{double} rayon\_fluor(fluor.get_rayon());
00189     \textcolor{keywordflow}{if} (nbr\_fluor != 0) \{ change\_epsilon(rayon\_fluor); \}
00190     \textcolor{keywordflow}{if} (canon) \{
00191         \textcolor{keywordflow}{for}(\textcolor{keywordtype}{int} i(1) ; i <= nbr\_fluor ; ++i)
00192         \{
00193             \textcolor{keywordtype}{double} pas(i*rayon\_fluor);
00194             ajouterParticule(\textcolor{keyword}{new} GLFluor(Vecteur(pas,pas,pas), Vecteur(250,250,250)));
00195         \}
00196     \} \textcolor{keywordflow}{else} \{
00197         \textcolor{keywordflow}{for}(\textcolor{keywordtype}{int} i(1) ; i <= nbr\_fluor ; ++i)
00198         \{ ajouterParticule(\textcolor{keyword}{new} GLFluor(enceinte, tirage, temperature)); \}
00199     \}
00200 \}
00201 
00206 \textcolor{keywordtype}{void} Systeme::ajouterHelium(\textcolor{keywordtype}{unsigned} \textcolor{keywordtype}{int} nbr\_helium, \textcolor{keywordtype}{bool} canon)
00207 \{
00208     GLHelium helium(Vecteur(0,0,0), Vecteur(0,0,0));
00209     \textcolor{keywordtype}{double} rayon\_helium(helium.get_rayon());
00210     \textcolor{keywordflow}{if} (nbr\_helium != 0) \{ change\_epsilon(rayon\_helium); \}
00211     \textcolor{keywordflow}{if} (canon) \{
00212         \textcolor{keywordflow}{for}(\textcolor{keywordtype}{int} i(1) ; i <= nbr\_helium ; ++i)
00213         \{
00214             \textcolor{keywordtype}{double} pas(i*rayon\_helium);
00215             ajouterParticule(\textcolor{keyword}{new} GLHelium(Vecteur(pas,pas,pas), Vecteur(250,250,250)));
00216         \}
00217     \} \textcolor{keywordflow}{else} \{
00218         \textcolor{keywordflow}{for}(\textcolor{keywordtype}{int} i(1) ; i <= nbr\_helium ; ++i)
00219         \{ ajouterParticule(\textcolor{keyword}{new} GLHelium(enceinte, tirage, temperature)); \}
00220     \}
00221 \}
00222 
00227 \textcolor{keywordtype}{void} Systeme::ajouterNeon(\textcolor{keywordtype}{unsigned} \textcolor{keywordtype}{int} nbr\_neon, \textcolor{keywordtype}{bool} canon)
00228 \{
00229     GLNeon neon(Vecteur(0,0,0), Vecteur(0,0,0));
00230     \textcolor{keywordtype}{double} rayon\_neon(neon.get_rayon());
00231     \textcolor{keywordflow}{if} (nbr\_neon != 0) \{ change\_epsilon(rayon\_neon); \}
00232     \textcolor{keywordflow}{if} (canon) \{
00233         \textcolor{keywordflow}{for}(\textcolor{keywordtype}{int} i(1) ; i <= nbr\_neon ; ++i)
00234         \{
00235             \textcolor{keywordtype}{double} pas(i*rayon\_neon);
00236             ajouterParticule(\textcolor{keyword}{new} GLNeon(Vecteur(pas,pas,pas), Vecteur(250,250,250)));
00237         \}
00238     \} \textcolor{keywordflow}{else} \{
00239         \textcolor{keywordflow}{for}(\textcolor{keywordtype}{int} i(1) ; i <= nbr\_neon ; ++i)
00240         \{ ajouterParticule(\textcolor{keyword}{new} GLNeon(enceinte, tirage, temperature)); \}
00241     \}
00242 \}
00243 
00248 \textcolor{keywordtype}{void} Systeme::initialise(\textcolor{keywordtype}{unsigned} \textcolor{keywordtype}{int} nbr\_argon, \textcolor{keywordtype}{unsigned} \textcolor{keywordtype}{int} nbr\_fluor,
00249                          \textcolor{keywordtype}{unsigned} \textcolor{keywordtype}{int} nbr\_helium, \textcolor{keywordtype}{unsigned} \textcolor{keywordtype}{int} nbr\_neon)
00250 \{
00251     epsilon = enceinte.getLargeur();
00252     ajouterArgon(nbr\_argon);
00253     ajouterFluor(nbr\_fluor);
00254     ajouterHelium(nbr\_helium);
00255     ajouterNeon(nbr\_neon);
00256 \}
00257 
00259 \textcolor{keywordtype}{void} Systeme::initialise()
00260 \{
00261     initialise(tirage.uniforme_entier(0, 75),
00262                tirage.uniforme_entier(0, 3),
00263                tirage.uniforme_entier(0, 100),
00264                tirage.uniforme_entier(0, 75)); 
00265 \}
00266 
00269 \textcolor{keywordtype}{void} Systeme::change\_epsilon(\textcolor{keywordtype}{double} rayon)
00270 \{ \textcolor{keywordflow}{if} (epsilon > rayon) \{ epsilon = rayon; \}\}
\end{DoxyCode}
