\section{Particule.\+cc}
\label{_particule_8cc_source}\index{/\+Users/burkhard/\+Dropbox/projet d'info/maximilian-\/g065/\+Simulation d'un gaz parfait/\+Particule.\+cc@{/\+Users/burkhard/\+Dropbox/projet d'info/maximilian-\/g065/\+Simulation d'un gaz parfait/\+Particule.\+cc}}

\begin{DoxyCode}
00001 
00009 \textcolor{preprocessor}{#include "Particule.h"}
00010 \textcolor{preprocessor}{#include <cmath>}
00011 
00012 \textcolor{keyword}{using namespace }std;
00013 
00014 \textcolor{comment}{/*================================================================================}
00015 \textcolor{comment}{ * Definition des constructeurs}
00016 \textcolor{comment}{ *================================================================================*/}
00018 Particule::Particule()
00019     : position(Vecteur(0,0,0)), vitesse(Vecteur(0,0,0)), masse(0), rayon(0) \{\}
00020 
00027 Particule::Particule(Vecteur position, Vecteur vitesse, \textcolor{keywordtype}{double} masse, \textcolor{keywordtype}{double} rayon)
00028     : position(position), vitesse(vitesse), masse(masse), rayon(rayon) \{\}
00029 
00033 Particule::Particule(Enceinte \textcolor{keyword}{const}& enceinte, GenerateurAleatoire& tirage, \textcolor{keywordtype}{double} temperature, \textcolor{keywordtype}{double} 
      masse, \textcolor{keywordtype}{double} rayon)
00034     : position(Vecteur(tirage.uniforme\_reel(0, enceinte.getLargeur()),
00035                        tirage.uniforme\_reel(0, enceinte.getLongueur()),
00036                        tirage.uniforme\_reel(0, enceinte.getHauteur()))), 
00037       masse(masse), rayon(rayon)
00038 \{
00039     \textcolor{keywordtype}{double} ecart\_type (sqrt(8314.472 * temperature / masse));
00040     vitesse = Vecteur(tirage.gaussienne(0, ecart\_type),
00041                       tirage.gaussienne(0, ecart\_type),
00042                       tirage.gaussienne(0, ecart\_type));
00043 \}                     
00044 
00045 \textcolor{comment}{/*================================================================================}
00046 \textcolor{comment}{ * Definition des méthodes}
00047 \textcolor{comment}{ *================================================================================*/}
00051 Vecteur Particule::getPosition()\textcolor{keyword}{ const}
00052 \textcolor{keyword}{}\{
00053     \textcolor{keywordflow}{return} position;
00054 \}
00057 ostream& Particule::afficher(ostream& sortie)\textcolor{keyword}{ const}
00058 \textcolor{keyword}{}\{
00059     \textcolor{keywordflow}{return} sortie << \textcolor{stringliteral}{"pos : "} << position << \textcolor{stringliteral}{" ; v : "} << vitesse
00060                   << \textcolor{stringliteral}{" ; m : "} << masse;
00061 \}
00062 
00064 \textcolor{keywordtype}{void} Particule::evolue(\textcolor{keywordtype}{double} dt)
00065 \{
00066     position += vitesse * dt;
00067 \}
00068 
00073 \textcolor{keywordtype}{void} Particule::gere_sorties(Enceinte \textcolor{keyword}{const}& enceinte)
00074 \{
00075     \textcolor{comment}{// Par rapport a X}
00076     \textcolor{keywordflow}{while} (position.getX() - rayon < 0 or position.getX() + rayon > enceinte.
      getLargeur()) \{
00077         \textcolor{keywordflow}{if} (position.getX() - rayon < 0)
00078         \{
00079             position = Vecteur (-(position.getX() - rayon) + rayon, position.
      getY(), position.getZ());
00080             vitesse = Vecteur (-vitesse.getX(), vitesse.getY(), vitesse.getZ());
00081             \textcolor{comment}{// Rebond sur la face 1}
00082         \} \textcolor{keywordflow}{else}
00083         \{
00084             position = Vecteur (enceinte.getLargeur() - (position.getX() + rayon - enceinte.
      getLargeur()) - rayon, position.getY(), position.getZ());
00085             vitesse = Vecteur (-vitesse.getX(), vitesse.getY(), vitesse.getZ());
00086             \textcolor{comment}{// Rebond sur la face 2}
00087         \}
00088     \}
00089  
00090     \textcolor{comment}{// Par rapport a Y}
00091     \textcolor{keywordflow}{while} (position.getY() - rayon < 0 or position.getY() + rayon > enceinte.
      getLongueur()) \{
00092         \textcolor{keywordflow}{if} (position.getY() - rayon < 0)
00093         \{
00094             position = Vecteur (position.getX(), -(position.getY() - rayon) + 
      rayon, position.getZ());
00095             vitesse = Vecteur (vitesse.getX(), -vitesse.getY(), vitesse.getZ());
00096             \textcolor{comment}{// Rebond sur la face 3}
00097         \} \textcolor{keywordflow}{else}
00098         \{
00099             position = Vecteur (position.getX(), enceinte.getLongueur() - (
      position.getY() + rayon - enceinte.getLongueur()) - rayon, position.getZ());
00100             vitesse = Vecteur (vitesse.getX(), -vitesse.getY(), vitesse.getZ());
00101             \textcolor{comment}{// Rebond sur la face 4}
00102         \}
00103     \}
00104     
00105     \textcolor{comment}{// Par rapport a Z}
00106     \textcolor{keywordflow}{while} (position.getZ() - rayon < 0 or position.getZ() + rayon > enceinte.
      getHauteur()) \{
00107         \textcolor{keywordflow}{if} (position.getZ() - rayon < 0)
00108         \{
00109             position = Vecteur (position.getX(), position.getY(), -(position.
      getZ() - rayon) + rayon);
00110             vitesse = Vecteur (vitesse.getX(), vitesse.getY(), -vitesse.getZ());
00111             \textcolor{comment}{// Rebond sur la face 5}
00112         \} \textcolor{keywordflow}{else}
00113         \{
00114             position = Vecteur (position.getX(), position.getY(), enceinte.
      getHauteur() - (position.getZ() + rayon - enceinte.getHauteur()) - rayon);
00115             vitesse = Vecteur (vitesse.getX(), vitesse.getY(), -vitesse.getZ());
00116             \textcolor{comment}{// Rebond sur la face 6}
00117         \}
00118     \}
00119 \}
00120 
00124 Vecteur Particule::pavageCubique(\textcolor{keywordtype}{double} epsilon)\textcolor{keyword}{ const}
00125 \textcolor{keyword}{}\{ \textcolor{keywordflow}{return} Vecteur (floor(position.getX()*(1/epsilon)),
00126                   floor(position.getY()*(1/epsilon)),
00127                   floor(position.getZ()*(1/epsilon))); \}
00128 
00132 \textcolor{keywordtype}{void} Particule::collision(Particule& p, GenerateurAleatoire& tirage)
00133 \{
00134     Vecteur vg (masse/(masse + p.masse) * vitesse + p.masse/(masse + p.masse) * p.
      vitesse);
00135     
00136     \textcolor{keywordtype}{double} L (sqrt((vitesse-vg) * (vitesse-vg))); \textcolor{comment}{// racine carrée du produit scalaire d'un vecteur avec
       lui-même = norme de ce vecteur}
00137     
00138     \textcolor{keywordtype}{double} z(tirage.uniforme_reel(-L, L));
00139     \textcolor{keywordtype}{double} phi(tirage.uniforme_reel(0, 2 * M\_PI));
00140     
00141     \textcolor{keywordtype}{double} r (sqrt(L * L - z * z));
00142     Vecteur v0 (r * cos(phi), r * sin(phi), z);
00143     
00144     vitesse = vg + v0;
00145     p.vitesse = vg - (masse/p.masse) * v0;
00146 \}
00147 
00150 \textcolor{keywordtype}{double} Particule::get_rayon() 
00151 \{ \textcolor{keywordflow}{return} rayon; \}
00152 
00153 \textcolor{comment}{/*================================================================================}
00154 \textcolor{comment}{ * Definition des fonctions}
00155 \textcolor{comment}{ *================================================================================*/}
00156 
00160 ostream& operator<<(ostream& sortie, Particule \textcolor{keyword}{const}& p)
00161 \{
00162     \textcolor{keywordflow}{return} p.afficher(sortie);
00163 \}
00164 
\end{DoxyCode}
