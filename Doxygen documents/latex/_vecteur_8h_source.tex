\section{Vecteur.\+h}
\label{_vecteur_8h_source}\index{/\+Users/burkhard/\+Dropbox/projet d'info/maximilian-\/g065/\+Simulation d'un gaz parfait/\+Vecteur.\+h@{/\+Users/burkhard/\+Dropbox/projet d'info/maximilian-\/g065/\+Simulation d'un gaz parfait/\+Vecteur.\+h}}

\begin{DoxyCode}
00001 
00010 \textcolor{preprocessor}{#ifndef PRJ\_VECTEUR\_H}
00011 \textcolor{preprocessor}{#define PRJ\_VECTEUR\_H}
00012 
00013 \textcolor{preprocessor}{#include <iostream>}
00014 
00015 
00023 \textcolor{keyword}{class }Vecteur
00024 \{
00025     \textcolor{keyword}{private} :
00026 \textcolor{comment}{/*================================================================================}
00027 \textcolor{comment}{ * Definition des attributs}
00028 \textcolor{comment}{ *================================================================================*/}
00029     \textcolor{keywordtype}{double} x;
00030     \textcolor{keywordtype}{double} y;
00031     \textcolor{keywordtype}{double} z;
00032     
00033     \textcolor{keyword}{public}:
00034 \textcolor{comment}{/*================================================================================}
00035 \textcolor{comment}{ * Prototyopes des constructeurs}
00036 \textcolor{comment}{ *================================================================================*/}
00038     Vecteur();
00040     Vecteur(\textcolor{keywordtype}{double} x, \textcolor{keywordtype}{double} y, \textcolor{keywordtype}{double} z);
00041     
00042 \textcolor{comment}{/*================================================================================}
00043 \textcolor{comment}{ * Prototypes des methodes}
00044 \textcolor{comment}{ *================================================================================*/}
00046     \textcolor{keywordtype}{double} getX() \textcolor{keyword}{const};
00048     \textcolor{keywordtype}{double} getY() \textcolor{keyword}{const};
00050     \textcolor{keywordtype}{double} getZ() \textcolor{keyword}{const};
00052     \textcolor{keywordtype}{bool} operator==(Vecteur \textcolor{keyword}{const}&) \textcolor{keyword}{const};
00054     \textcolor{keywordtype}{bool} operator!=(Vecteur \textcolor{keyword}{const}& v) \textcolor{keyword}{const};
00056     Vecteur& operator+=(Vecteur \textcolor{keyword}{const}& v1);
00058     Vecteur& operator-=(Vecteur \textcolor{keyword}{const}& v1);
00060     Vecteur& operator*=(\textcolor{keywordtype}{double} \textcolor{keyword}{const}& scalaire);    
00062     \textcolor{keywordtype}{double} operator*(\textcolor{keyword}{const} Vecteur& v1) \textcolor{keyword}{const};
00064     Vecteur& operator^=(Vecteur \textcolor{keyword}{const}& v1);\textcolor{comment}{// ATTENTION XOR a une plus grande priorité à cause du ou
       exclusif donc toujours mettre des parentaises entre !!!!!}
00066 \textcolor{comment}{}    \textcolor{keyword}{const} Vecteur operator-(); 
00068     std::ostream& afficher(std::ostream& sortie) \textcolor{keyword}{const};
00069 
00070 \};
00071 
00072 \textcolor{comment}{/*================================================================================}
00073 \textcolor{comment}{ * Prototypes des fonctions}
00074 \textcolor{comment}{ *================================================================================*/}
00076 std::ostream& operator<<(std::ostream& sortie, Vecteur \textcolor{keyword}{const}& v1);
00078 \textcolor{keyword}{const} Vecteur operator+(Vecteur v1, Vecteur \textcolor{keyword}{const}& v2);
00080 \textcolor{keyword}{const} Vecteur operator-(Vecteur v1, Vecteur \textcolor{keyword}{const}& v2);
00082 \textcolor{keyword}{const} Vecteur operator*(\textcolor{keywordtype}{double} scalaire, Vecteur v1);
00084 \textcolor{keyword}{const} Vecteur operator*(Vecteur v1, \textcolor{keywordtype}{double} scalaire);
00086 \textcolor{keyword}{const} Vecteur operator^(Vecteur v1, Vecteur \textcolor{keyword}{const}& v2);\textcolor{comment}{// ATTENTION XOR a une plus grande priorité à cause
       du ou exclusif donc toujours mettre des parentaises entre !!!!!}
00087 
00088 \textcolor{preprocessor}{#endif // PRJ\_VECTEUR\_H}
\end{DoxyCode}
