\section{exercice\+P9.\+cc}
\label{exercice_p9_8cc_source}\index{/\+Users/burkhard/\+Dropbox/\+Projet d'info/\+Approche\+Fin/exercice\+P9.\+cc@{/\+Users/burkhard/\+Dropbox/\+Projet d'info/\+Approche\+Fin/exercice\+P9.\+cc}}

\begin{DoxyCode}
00001 \textcolor{preprocessor}{#include <iostream>}
00002 \textcolor{preprocessor}{#include "Systeme.h"}
00003 \textcolor{preprocessor}{#include "TXTHelium.h"}
00004 \textcolor{preprocessor}{#include "TXTNeon.h"}
00005 \textcolor{preprocessor}{#include "TXTArgon.h"}
00006 
00007 \textcolor{keywordtype}{int} main() \{
00008     
00009     Systeme john;
00010     john.ajouterParticule(\textcolor{keyword}{new} TXTHelium(Vecteur(1,1,1), Vecteur(0,0,0)));
00011     john.ajouterParticule(\textcolor{keyword}{new} TXTNeon(Vecteur(1,18.5,1), Vecteur(0,0.2, 0)));
00012     john.ajouterParticule(\textcolor{keyword}{new} TXTArgon(Vecteur(1,1,3.1), Vecteur(0,0,-0.5)));
00013     
00014     cout << \textcolor{stringliteral}{"Le système est constitué des trois particules suivantes :"} << endl;
00015     
00016     john.dessine();
00017     
00018     cout << \textcolor{stringliteral}{"Lancement de la simulation :"} << endl
00019          << \textcolor{stringliteral}{"==========----------"} << endl;
00020     
00021     \textcolor{keywordflow}{for} (\textcolor{keywordtype}{int} i(1) ; i <= 10 ; ++i)
00022     \{
00023         john.evolue(1, \textcolor{keyword}{true});
00024         john.dessine();
00025         cout << \textcolor{stringliteral}{"==========----------"} << endl;
00026     \}
00027     
00028 \}
\end{DoxyCode}
