\section{Vecteur.\+cc}
\label{_vecteur_8cc_source}\index{/\+Users/burkhard/\+Dropbox/projet d'info/maximilian-\/g065/\+Simulation d'un gaz parfait/\+Vecteur.\+cc@{/\+Users/burkhard/\+Dropbox/projet d'info/maximilian-\/g065/\+Simulation d'un gaz parfait/\+Vecteur.\+cc}}

\begin{DoxyCode}
00001 
00009 \textcolor{preprocessor}{#include "Vecteur.h"}
00010 
00011 \textcolor{keyword}{using namespace }std;
00012 
00013 \textcolor{comment}{/*================================================================================}
00014 \textcolor{comment}{ * Definition des constructeurs}
00015 \textcolor{comment}{ *================================================================================*/}
00020 Vecteur::Vecteur()
00021     : x(0), y(0), z(0) \{\}
00022 
00030 Vecteur::Vecteur(\textcolor{keywordtype}{double} x, \textcolor{keywordtype}{double} y, \textcolor{keywordtype}{double} z)
00031     : x(x), y(y), z(z) \{\}
00032 
00033 \textcolor{comment}{/*================================================================================}
00034 \textcolor{comment}{ * Definition des méthodes }
00035 \textcolor{comment}{ *================================================================================*/}
00041 \textcolor{keywordtype}{double} Vecteur::getX()\textcolor{keyword}{ const}
00042 \textcolor{keyword}{}\{
00043     \textcolor{keywordflow}{return} x;
00044 \} 
00050 \textcolor{keywordtype}{double} Vecteur::getY()\textcolor{keyword}{ const}
00051 \textcolor{keyword}{}\{
00052     \textcolor{keywordflow}{return} y;
00053 \}
00059 \textcolor{keywordtype}{double} Vecteur::getZ()\textcolor{keyword}{ const}
00060 \textcolor{keyword}{}\{
00061     \textcolor{keywordflow}{return} z;
00062 \}
00063 
00068 \textcolor{keywordtype}{bool} Vecteur::operator==(Vecteur \textcolor{keyword}{const}& v)\textcolor{keyword}{ const}
00069 \textcolor{keyword}{}\{
00070     \textcolor{keywordflow}{if} (x == v.x and y == v.y and z == v.z) 
00071     \{ 
00072         \textcolor{keywordflow}{return} \textcolor{keyword}{true}; 
00073     \}\textcolor{keywordflow}{else} 
00074     \{ 
00075         \textcolor{keywordflow}{return} \textcolor{keyword}{false}; 
00076     \}
00077 \}
00078 
00084 \textcolor{keywordtype}{bool} Vecteur::operator!=(Vecteur \textcolor{keyword}{const}& v)\textcolor{keyword}{ const}
00085 \textcolor{keyword}{}\{
00086     \textcolor{keywordflow}{return} not(*\textcolor{keyword}{this} == v);
00087 \}
00088 
00095 Vecteur& Vecteur::operator+=(Vecteur \textcolor{keyword}{const}& v1)
00096 \{
00097     x += v1.x;
00098     y += v1.y;
00099     z += v1.z;
00100     \textcolor{keywordflow}{return} *\textcolor{keyword}{this};
00101 \}
00102 
00109 Vecteur& Vecteur::operator-=(Vecteur \textcolor{keyword}{const}& v1) 
00110 \{
00111     x -= v1.x;
00112     y -= v1.y;
00113     z -= v1.z;
00114     \textcolor{keywordflow}{return} *\textcolor{keyword}{this};
00115 \}
00116 
00122 \textcolor{keyword}{const} Vecteur Vecteur::operator-()
00123 \{
00124     \textcolor{keywordflow}{return} ((*\textcolor{keyword}{this})*= (-1.));
00125 \}
00126 
00133 Vecteur& Vecteur::operator*=(\textcolor{keywordtype}{double} \textcolor{keyword}{const}& scalaire)
00134 \{
00135     x *= scalaire;
00136     y *= scalaire;
00137     z *= scalaire;
00138     \textcolor{keywordflow}{return} *\textcolor{keyword}{this};
00139 \}   
00140 
00148 Vecteur& Vecteur::operator^=(Vecteur \textcolor{keyword}{const}& v1)
00149 \{
00150     \textcolor{keywordtype}{double} x\_(x), y\_(y);
00151     x = (y * v1.z - z * v1.y);
00152     y = (z * v1.x - x\_ * v1.z);
00153     z = (x\_ * v1.y - y\_ * v1.x);
00154     \textcolor{keywordflow}{return} *\textcolor{keyword}{this};
00155 \}
00156 
00163 \textcolor{keywordtype}{double} Vecteur::operator*(\textcolor{keyword}{const} Vecteur& v1)\textcolor{keyword}{ const}
00164 \textcolor{keyword}{}\{
00165     \textcolor{keywordflow}{return} (v1.x*x + v1.y*y + v1.z*z);
00166 \}
00167 
00172 ostream& Vecteur::afficher(ostream& sortie)\textcolor{keyword}{ const}
00173 \textcolor{keyword}{}\{
00174     \textcolor{keywordflow}{return} sortie << x << \textcolor{stringliteral}{" "} << y << \textcolor{stringliteral}{" "} << z;
00175 \}
00176 
00177 \textcolor{comment}{/*================================================================================}
00178 \textcolor{comment}{ * Definition des fonctions }
00179 \textcolor{comment}{ *================================================================================*/}
00180 
00181 
00190 ostream& operator<<(ostream& sortie, Vecteur \textcolor{keyword}{const}& v1)
00191 \{
00192     \textcolor{keywordflow}{return} v1.afficher(sortie);
00193 \}
00194 
00195 \textcolor{comment}{/* Les méthodes suivantes sont écrites sur 2 lignes pour éviter la copie}
00196 \textcolor{comment}{ * inutile faîtes par certain compilateurs }
00197 \textcolor{comment}{ */}
00198 
00204 \textcolor{keyword}{const} Vecteur operator+(Vecteur v1, Vecteur \textcolor{keyword}{const}& v2) 
00205 \{
00206     v1 += v2;
00207     \textcolor{keywordflow}{return} v1;
00208 \}
00209 
00215 \textcolor{keyword}{const} Vecteur operator-(Vecteur v1, Vecteur \textcolor{keyword}{const}& v2) 
00216 \{
00217     v1 -= v2;
00218     \textcolor{keywordflow}{return} v1;
00219 \}
00220 
00228 \textcolor{keyword}{const} Vecteur operator*(\textcolor{keywordtype}{double} scalaire, Vecteur v1)
00229 \{
00230     v1 *= scalaire;
00231     \textcolor{keywordflow}{return} v1;
00232 \}
00233 
00239 \textcolor{keyword}{const} Vecteur operator*(Vecteur v1, \textcolor{keywordtype}{double} scalaire)
00240 \{
00241     v1 *= scalaire;
00242     \textcolor{keywordflow}{return} v1;
00243 \}
00244 
00252 \textcolor{keyword}{const} Vecteur operator^(Vecteur v1, Vecteur \textcolor{keyword}{const}& v2)
00253 \{
00254     v1 ^= v2;
00255     \textcolor{keywordflow}{return} v1;
00256 \}
\end{DoxyCode}
